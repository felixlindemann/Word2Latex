\usepackage[% 
colorinlistoftodos,% 
%disable% wirkt sich auch auf die "Überarbeiten"-Befehle aus! 
]{todonotes} 

%------------ 
% Neue Befehle zum Überarbeiten von Text à la Word: 
% 
% Ersetzen-Funktion \replace{alter Text}{neuer Text}{Anmerkung/Kommentarnummer} 
\makeatletter 
\if@todonotes@disabled% 
\newcommand{\replace}[3]{#2} 
\else% \if@todonotes@disabled 
\newcommand{\replace}[3]{% 
   \textcolor{blue}{% 
      #2 %NEU; Das Leerzeichen stimmt hier! Sonst klebt der Text von neu an alt. 
      \sout{#1}%ALT durchgestrichen 
      \todo[linecolor=blue, backgroundcolor=blue!10,bordercolor=blue]{\##3}% 
   }% 
} 
\fi 
\makeatother 

% Einfügen-Funktion \add{eingefügter Text}{Anmerkung/Kommentarnummer} 
\makeatletter 
\if@todonotes@disabled% 
\newcommand{\add}[2]{#1} 
\else% \if@todonotes@disabled 
\newcommand{\add}[2]{% 
   \textcolor{red}{% 
      #1% 
      \todo[linecolor=red, backgroundcolor=red!10,bordercolor=red]{\##2}% 
   }% 
} 
\fi 
\makeatother 

% Kommentarfunktion \comment{Zu Kommentierender Text}{Kommentar}{Anmerkung/Kommentarnummer} 
\makeatletter 
\if@todonotes@disabled% 
\newcommand{\comment}[3]{#1} 
\else% \if@todonotes@disabled 
\newcommand{\comment}[3]{% 
   \textcolor{orange}{% 
      #1% 
      \todo[linecolor=orange, backgroundcolor=orange!10,bordercolor=orange]{\##3: #2}% 
   }% 
} 
\fi 
\makeatother 

% Kommentarfunktion \wiggle{Zu unterschlängelnder Text}{Anmerkung/Kommentarnummer} 
\makeatletter 
\if@todonotes@disabled% 
\newcommand{\wiggle}[3]{#1} 
\else% \if@todonotes@disabled 
\newcommand{\wiggle}[3]{% 
   \textcolor{green}{% 
      \uwave{#1}% 
      \todo[linecolor=green, backgroundcolor=green!10,bordercolor=green]{\##3: #2}% 
   }% 
} 
\fi 
\makeatother 
%------------ 